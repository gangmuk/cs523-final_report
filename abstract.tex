\begin{abstract}

    Many of modern large systems are composed of a group of individual components. Each component is doing their own dedicated duty based on what it observes from the environment. There are different programming models but we will focus on one where each component keeps checking on the status of the system and tries to make the system converge on a certain desirable status. There is no a single global controller who has the perfect knowledge of the entire system and guarantees that it will not fall into undesirable status. Not just because they are such a big and complex system but there are some particular reasons facilitating the system to be unintentionally faulty. First, multiple components often interact each other and control the same part of the system. Inherently it is possible that they behave contradictorily each other. Seocnd, each component is often developed by multiple people, multiple teams and even multiple organizations. Each team cannot understand how other parts of the system functions in detail. It is almost infeasible and not practical for each component to perfectly take into consideration other parts of the system. This paper did case study to understand the failures involving multiple components. We pick Kubernetes container orchestration system as a representative example to dive into more specific failure cases. We analyzed 10 different failure cases which were reported in Kubernetes github issues, Kubecon (Kubernetes conference), and blog posts.
    
\end{abstract}
