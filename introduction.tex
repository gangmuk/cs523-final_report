\section{Introduction}
\label{sec:introduction}

Modern cluster management system is composed of a collection of control components. Each controller directly or indirectly interacts with each other but each one has its own execution routine which is not confined to other controller. Each of them functions independently and it is not a subset of another controller.

There is no a single global controller who has the perfect knowledge of the entire system. Instead each controller is distributed and fulfills its own dedicated duty individually, hoping the entire cluster to converges on desirable status eventually. By nature of distributed controller design, however, there is no guarantee that the cluster will not fall into undesirable status. 

Cluster management system which has designed in a distributed controller Not just because they are such a big and complex system but there are more concrete reasons facilitating the system to be unintentionally faulty. First, multiple components often interact each other and control the same part of the system. Inherently it is possible that they behave contradictorily each other. Seocnd, each component is often developed by multiple people, multiple teams and even multiple organizations. Each team cannot understand how other parts of the system functions in detail. It is almost infeasible and not practical for each component to perfectly take into consideration other parts of the system. This paper did case study to understand the failures involving multiple components. We pick Kubernetes container orchestration system as a representative example to dive into more specific failure cases. We analyzed 10 different failure cases which were reported in Kubernetes github issues, Kubecon (Kubernetes conference), and blog posts.
