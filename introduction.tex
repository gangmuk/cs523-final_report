\section{Introduction}
\label{sec:introduction}
Modern cluster management system is composed of a collection of loosely coupled control components. Each controller operates asynchronously in distributed fashion. They fulfill their own dedicated duties independently, hoping the entire cluster to converges on desirable status eventually. There is no single global controller who has the perfect knowledge of the entire system. By nature of distributed controller design, however, there is no guarantee that the entire cluster will not fall into undesirable status such as oscillating phenomenon. 

This is not just because it is such a huge and complex system but there are two concrete reasons facilitating the system to be unintentionally unstable. One is the design of the system and the other is the development environment. First, control realms of multiple controllers in the cluster sometimes are not evidently exclusvie but could be overlapped with each other. For example, scheduler and descheduler both are involved in pod scheduling. Inherently it is possible that they behave in contradictory way that configurations in multiple controllers diverge and cluster status never converges or go over sub-optimal path to stable status.
Seocnd, each controller is often developed by multiple people, multiple teams and even multiple organizations. Each team cannot understand how other parts of the system functions in detail. It is almost infeasible and not practical for each component to perfectly take into consideration other parts of the system. Covering all conditions that could be possibly induced by multiple controllers' interaction requires inhibitively large search space which is not practical and should not be pursued.

In this paper, we will deeply understand how multiple controllers in modern cluster management system can cause pathological or non-optimal behavior. Further, we will provide some insight and promising solution for this problem. To dive into more specific failure cases, we pick Kubernetes container orchestration system as a representative example which is the most widely used cluster management system. We analyzed and reproduced 10 different failure cases which were reported in Kubernetes github issues, Kubecon (Kubernetes conference), and blog posts.
